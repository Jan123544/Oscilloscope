\documentclass[main.tex]{subfiles}
\begin{document}
	\section{Program}
		\begin{multicols*}{2}
			\noindent Program je tvorený 3 stavovými automatmi, ktorých činnosť je previazaná systémom udalostí. Každý stavový automat má štruktúru udalostí s  príznakmi, ktoré sa nastavia v prípade, že udalosť nastane. V hlavnej slučke programu, sa vykonáva aktualizácia jednotlivých stavových automatov. Túto aktualizáciu vykonáva funkcia *\_update, kde $*$ môže byť buď OSCI\_transceiver (príjmač a vysielač) alebo OSCI\_channel\_update (stavový automat riadiaci činnosť kanála). V krátkosti opíšeme činnosť týchto stavových automatov, pre detaily je vhodné konzultovať zdrojový kód. %Schematické znázornenie programu je na konci tejto kapitoly na \cref{fig:programClekovaSchema}.
			
			\subsection{Stavový automat vysielač/príjmač}
			Úlohou tohto stavového automatu, je reakcia na príchod nových dát, distribúcia nových nastavení a posielanie nových meraní do užívateľského prostredia. Jednoduché zobrazenie môžeme vidieť na \cref{fig:stavovyStrojVysielacPrijmac}.
			
			
			Počiatočný stav stavového automatu je $IDL-IDLE$, v tomto stave čaká na požiadavky z GUI, alebo na požiadavku od stavových automatov kanálov, na zaslanie nových dát.
			
			Cez perifériu UART sa pri požiadavke z GUI príjmu nové nastavenia. Prerušenie DMA TC pri prijatí, celej štruktúry nastavení, nastaví príznak stavovému automatu vysielač/príjmač. Tento príznak zodpovedá požiadavke na vykonanie merania (RME), načo stavový automat prejde do stavu SSH - Shutting down channels. V stave SSH nastaví príznaky kanálom, ktoré reagujú ukončením meraní a prejdú do stavu SHD - shutdown. Vysielač/prijmač čaká, až kým sa kanály nedostanú do stavu SHD v stave WSH - waiting for channels to shutdown. Potom prejde do stavu RCF - reconfiguring channels. V tomto stave sa zmenia parametre transformácie dát, kanály dostanú nové nastavenia a stavový automat prepne do stavu SCH - starting channels. V tomto novom stave sa nastaví príznak stavovým automatom kanálov, aby začali kanály monitorovať, čo znamená, že sa AD prevodníky spustia v Continuous stave a čaká sa na prerušenie Watchdogu. Nakoniec sa vysielač/príjmač vráti do stavu IDLE, v ktorom zostane pokiaľ nedostane ďalšiu požiadavku na meranie alebo iný typ požiadavky.
			
			V prípade, že sa vykoná meranie, stavový automat daného kanála, nastaví príznak stavovému automatu vysielač/príjmač, aby vykonal transformáciu nameraných dát a poslal tieto nové dáta do GUI. Stavový automat vysielač/príjmač, reaguje na tento príznak v stave IDLE a prejde do stavu GTS - gathering transforming and sending, v tomto stave sa vykoná už spomínaná transformácia dát a ich zaslanie do GUI. Vysielač/príjmač sa nakoniec vráti do stavu IDLE potom čo vyriešil všetky požiadavky na odoslanie dát, keďže tie sú rozdielne pre X a Y kanál.
			
			Požiadavka od GUI, môže byť aj požiadavkou o vykonanie transformácie, v prípade ktorej, sa zmenia parametre priamo v stave IDLE a vygeneruje sa požiadavka na zaslanie starého merania s novou transformáciou.
			
			
			\subfile{stavovyStrojVysielacPrijmac}
			\vskip 0.25cm
			
			Ešte pred meraním, je potrebné vytvoriť spojenie, ktoré slúži na overenie nahraného programu. V takomto prípade, je vysielač/príjmač v stave IDLE, z ktorého rovno pošle PONG bez prechodu do iného stavu.
			
			\subsection{Stavové automaty kanálov}
			Úlohou týchto stavových automatov je zastavovanie a spúšťanie merania. Štruktúra je zobrazená na \cref{fig:stavovyStrojMeraciehoKanalu}.
			
			\subfile{stavovyStrojMeraciehoKanalu}
			\vskip 0.25cm
			
			
			Počiatočný stav týchto stavových automatov je stav SHD - shutdown. Pri nastavení príznaku začatia monitorovania kanála, čo nastavuje stavový automat vysielač/príjmač, sa nastaví a spustí AD prevodník v móde Conitnous, čo pravidelne vzorkuje kanál a automat prejde do stavu MON - monitoring. V tomto stave sa čaká na prerušenie Watchdogu. Ktoré sa vyvolá, pokiaľ je napätie na kanály nad stanovenou hranicou, nastavenou v GUI. V tomto prerušení, sa hneď spúšťa meranie a automat prechádza do stavu MES - measuring. V stave MES ostáva, pokiaľ sa meranie nedokončí, čo sa určí na základe TC prerušenia DMA. V tomto prerušení sa nastaví príznak measurement complete, načo stavový automat reaguje buď prechodom do stavu SHD a nastavením príznaku SRQ - send request stavového automatu vysielač/prijmač. V prípade jednorázového merania v tomto stave ostáva, ale v prípade Continous merania sa v prerušení Hold-off časovača kanál znova prevedie do stavu MON - monitoring. 
			
			\subsection{Hlavičkové súbory}
			Program sa skladá z niekoľkých zdrojových, hlavičkových a  .c súborov. Hlavičkové súbory slúžia ako miesto deklarácie funkcií, pričom všetky nebudeme opisovať, opíšeme len tie, v ktorých sa nachádzajú dôležité konštanty, premenné a dátové štruktúry.
			
			\textit{osci\_defines.h } - nachádzajú sa tu kalibračné faktory, počiatočné nastavenia, enumerácie opkódov správ do GUI, enumerácia kanálov, makrá na MIN a MAX, definícia TRUE a FALSE, niektoré definované maximálne hodnoty unsigned registrov, enumerácia typov stavového stroja.
			
			\textit{osci\_data\_structures.h} - tento súbor obsahuje definície dátových štruktúr stavových automatov a rôznych abstrakcií ako napríklad meranie kanálu, čo predstavuje postupnosť vzoriek, dátové štruktúry nastavení poslaných z GUI, nastavení kanálov, udalostí jednotlivých stavových automatov, dátovú štruktúru aplikácie a dátovú štruktúru opisujúce parametre prechodu stavového automatu.
			
			\textit{osci\_channel\_state\_machine.h} - definícia stavov automatu, deklarácia callbackov, init a update funkcií stavového automatu kanálu.
			
			\textit{osci\_transceiver.h} - definícia stavov automatu, deklarácia callbackov, init a update funkcií stavového automatu vysielač/prijmač.
			
			\subsection{.C Súbory}
			Okrem inicializačných súborov periférií a iných štandardných súborov vygenerovaných pomocou CubeMX sa tu nachádzajú nasledovné súbory.
			
			\textit{ osci.c} - obsahuje funkciu, ktorá alokuje pamäť a inicializuje dátové štruktúry aplikácie
			
			\textit{ main.c} - volanie inicializačných funkcií, nastavenie globálnej referencie pre prerušenia a spustenie nekonečného cyklu update funkcií stavových strojov
			
			\textit{ osci\_adc.c} - pomocné funkcie, na ovládanie AD prevodníka. 
			
			
			\textit{ osci\_state\_machine.c} - všeobecné funkcie pre stavové automaty
			
			\textit{ osci\_channel\_state\_machine.c} - implementácia stavového automatu meracieho kanálu. 
			
			\textit{ osci\_configurator.c} -  funkcie na výpočet a prepočet parametrov kanálov, teda časovačov a transformácie merania. 
			
			\textit{ osci\_dma.c} - funkcie na jednoduchšiu obsluhu dma. 
			
			\textit{ osci\_error.c} - vo všeobecnosti debugovacie funkcie.
			
			\textit{ osci\_timer.c} - pomocné funkcie na obsluhu časovačov. 
			
			\textit{ osci\_transceiver.c} - implementácia stavového automatu vysielač/prijmač.
			
			\textit{ osci\_transform.c} - aplikácia transformácie podľa vypočítaných parametrov.
			
			\subsection{Postup merania}
			Pre lepšiu názornosť funkcie programu, opíšeme ešte celkový reťazec udalostí, od stlačenia tlačidla merania až po vykreslenie nameraných dát. Pričom uvažujeme, že zariadenie je už pripojené, teda, že už prebehol ping-pong handshake.
			
			Po stlačený tlačidla \textit{MEASURE} sa prečítajú aktuálne nastavenia GUI, ktoré sa pošlú cez \textit{UART} linku. UART príjmač generuje DMA požiadavky až kým sa nepríjme celá štruktúra nastavení, v prípade čoho DMA vygeneruje TC prerušenie daného kanálu (6).  Obslužná funkcia prerušenia nastaví event stavovému automatu vysielač/prijmač. Ten pri ďalšom volaní svojej update funkcie, deteguje event a rozhodne sa čo ďalej, na základe typu správy, ktorú dostal. Keďže uvažujeme, že užívateľ stlačil tlačidlo \textit{MEASURE}, tak sa vykoná nasledujúca sekvencia udalostí. Stavovým automatom kanálov sa pošle event na vypnutie a prejde do stavu \textit{WAITING\_FOR\_SHUTDOWN}, stavový automat vysielač/prijmač počká na prechod stavových automatov kanálov do režimu \textit{SHUTDOWN} a prejde do stavu \textit{RECONFIGURING\_CHANNELS}, stavový automat vysielač/prijmač prestaví parametre kanálov a prejde do stavu \textit{STARTING\_CHANNELS} v tomto stave stavový automat vysielač/prijmač nastaví event stavovým automatomom kanálov, že sa majú previesť do stavu monitoring, stavové automatye kanálov sa prevedú do stavu monitoring, čo predstavuje nastavenie continuous módu AD prevodníkov, ktoré začnú vzorkovať kanály a nastavenie watchdogov na úrovne poslané z GUI, ak nastane prerušenie daného watchdogu, nastavia sa časovače, DMA, AD prevodník a spustí sa meranie, stavový automat kanálu prejde do stavu \textit{MEASURING}, po ukončení merania DMA generuje prerušenie TC, v ktorom sa pošle event stavovému automatuu vysielač/prijmač, že má poslať nové dáta, ten reaguje a prejde do stavu \textit{GATHERING\_TRANSFORMING\_SENDING} zo stavu \text{IDLE} a pošle označené dáta, keďže dáta kanálov sa posielajú zvlášť.
			
			\subsection{Postup v prípade automatického aktualizácie parametrov transformácie}
			 V prípade, že ide o požiadavku na zaslanie dát s inou transformáciou, teda o tzv. \text{ONLY\_TRANSFORM} typ správy, tak sa len prepočítajú parametre transformácií jednotlivých kanálov a registruje sa požiadavka na odoslanie oboch kanálov, stavový automat prejde do stavu \textit{GATHERING\_TRANSFORMING\_SENDING}, v ktorom sa vykoná transformácia posledného merania a aktivuje sa DMA kanál pripojený na vysielač UART-u, ktorý pošle jednotlivé transformované merania do GUI.
			 
			 
		\end{multicols*}
		%\subfile{programCelkovaSchema}
\end{document}