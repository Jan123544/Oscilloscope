\documentclass[main.tex]{subfiles}
\begin{document}
	\section{Program}
		\begin{multicols*}{2}
			\noindent Program je tovrený 3 stavovými strojmi, ktorých činnosť je previazaná systémom udalostí. Každý stavový stroj má vlajky udalostí, ktoré sa nastavia v prípade, že udalosť nastane. V hlavnej slučke programu, sa vykonáva aktualizácia jednotlivých stavových strojov. Túto aktualizáciu vykonáva funkcia $*\_update$, kde $*$ môže byť buď $Transceiver$ (príjmač a vysielač) alebo $ChannelStateMachine$ (stavový stroj riadiaci činnosť meracieho kanála). V krátkosti opíšeme činnosť týchto stavových strojov, pre detaily je vhodné konzultovať zdrojový kód. Schematické znázornenie programu je na konci tejto kapitoly na \cref{fig:programClekovaSchema}.
			
			\subsection{Stavový stroj vysielač/príjmač}
			Úlohou tohto stavového stroja, je reakcia na príchod nových dát, distribúcia nových nastavení a posielanie nových meraní do uživateľského prostredia. Jednoduché zobrazenie môžeme vidieť na \cref{fig:stavovyStrojVysielacPrijmac}.
			
			Po príchode dát cez $UART$ linku, sa na základe obsahu prijatej správy, vysielač rozhodne, či má poslať $pong$ správu, zmeniť transformáciu dát, alebo prekonfigurovať stavový stroj kanálu na iný typ merania. Prekonfigurovanie prebieha vypnutím meraní, prepísaním parametrov a prevodom stavových strojov kanálov do stavu $MONITORING$, v ktorom $watchdog$ obvody sledujú hodnoty prevedné AD prevodníkom. Ak tieto hodnoty prekročia definované limity spustí sa meranie. Po vykonaní merania nastane udalosť, na ktorú stavový stroj vysielač/príjmač reaguje zaslaním nových dát do uživateľského prostredia.
			
			\subfile{stavovyStrojVysielacPrijmac}
			\vskip 0.25cm
			
			\subsection{Stavový stroj meracieho kanála}
			Úlohou tohto stavového stroja je zastavovanie a spúšťanie merania. Štruktúra je zobrazená na \cref{fig:stavovyStrojMeraciehoKanalu}.

			Bufre nastavení nastavuje vysielač/príjmač. Po príchode požiadavky na meranie nastane udalosť, ktorá prevedie stavové stroje meracieho kanála do stavu $SHUTDOWN$, zastavením časovačov a prevodov AD prevodníkov. Vysielač/príjmač reaguje na $SHUTDOWN$ stav kanálov vo svojej stavovej slučke, a prevedie ich do stavu $MONITORING$. V stave $MONITORING$ kanále vzorkujú svoj vstup a $watchody$ porovnávajú prevedené hodnoty s nastavenými limitmi. V prípade, že sú limity určené pre žačatie merania prekročené, začne sa meranie a kanál prejde do stavu $MEASURING$. V prerušení od $DMA$ modulu, sa kanály prevedú znova $SHUTDOWN$ módu a nastane udalosť, na ktorú vysielač bude reagovať zaslaním nových dát do uživateľského prostredia.
			
			\subfile{stavovyStrojMeraciehoKanalu}
			\vskip 0.25cm
			
			
			\subsection{Hlavičkové súbory}
			Program sa skladá z niekoľkých zdrojových súborov, niekoľkých hlavičkových súborov a niekoľkých .c súborov. Hlavičkové súbory slúžia ako miesto deklarácie funkcií nebudeme opisovať, opíšeme len tie, v ktorých sa nachádzajú dôležité konštanty a premenné.
			
			\textit{osci\_defines.h } - nachádzajú sa tu kalibračné faktory, počiatočné nastavenia, enumerácie opkódov správ do GUI, enumerácia kanálov, makrá na MIN a MAX, definícia TRUE a FALSE, niektoré definované maximálne hodnoty unsigned registrov, enumerácia typov stavového stroja.
			
			\textit{osci\_data\_structures.h} - tento súbor obsahuje definície dátových štruktúr stavových strojov a rôznych abstrakcií ako napríklad meranie kanálu, čo predstavuje postupnosť vzoriek, dátové štruktúry nastavení poslaných z GUI, nastavení kanálov, udalostí jednotlivých stavových strojov, dátovú štruktúru aplikácie a dátovú štruktúru opisujúce parametre prechodu stavového stroja.
			\subsection{.C Súbory}
			Okrem inicializačných súborov periférií a iných štandardných súborov vygenerovaných pomocou CubeMX sa tu nachádzajú nasledovné súbory.
			
			\textit{ osci\_adc.h} - pomocné funkcie, na ovládanie AD prevodníka. 
			
			\textit{ osci\_channel\_state\_machine.c} - implementácia stavového stroja meracieho kanálu. 
			
			\textit{ osci\_configurator.c} - v tomto zdrojovom súbore sú funkcie na výpočet a prepočet parametrov kanálov, teda časovačov a transformácie merania. 
			
			\textit{ osci\_dma.c} - funkcie na jednoduchšiu obsluhu dma. 
			
			\textit{ osci\_error.c} - vo všeobecnosti debugovacie funkcie.
			
			\textit{ osci\_timer.c} - pomocné funkcie na obsluhu časovačov. 
			
			\textit{ osci\_transceiver.c} - implementácia stavového stroja vysielač/prijmač.
			
			\textit{ osci\_transform.c} - aplikácia transformácie podľa vypočítaných parametrov.
			
			\subsection{Postup merania}
			Pre lepšiu názornosť funkcie programu, opíšeme ešte celkový reťazec udalostí, od stlačenia tlačidla merania až po vykreslenie nameraných dát. Pričom uvažujeme, že zariadenie je už pripojené, teda, že už prebehol ping-pong handshake.
			
			Po stlačený tlačidla \textit{MEASURE} sa prečítajú aktuálne nastavenia GUI, ktoré sa pošlú cez \textit{UART} linku. UART príjmač generuje DMA požiadavky až kým sa nepríjme celá štruktúra nastavení, v prípade čoho DMA vygeneruje TC prerušenie daného kanálu (6).  Obslužná funkcia prerušenia nastaví event stavovému stroju vysielač/prijmač. Ten pri ďalšom volaní svojej update funkcie, deteguje event a rozhodne sa čo ďalej, na základe typu správy, ktorú dostal. Keďže uvažujeme, že užívateľ stlačil tlačidlo \textit{MEASURE}, tak sa vykoná nasledujúca sekvencia udalostí. Stavovým strojom kanálov sa pošle event na vypnutie a prejde do stavu \textit{WAITING\_FOR\_SHUTDOWN}, stavový stroj vysielač/prijmač počká na prechod stavových strojov kanálov do režimu \textit{SHUTDOWN} a prejde do stavu \textit{RECONFIGURING\_CHANNELS}, stavový stroj vysielač/prijmač prestaví parametre kanálov a prejde do stavu \textit{STARTING\_CHANNELS} v tomto stave stavový stroj vysielač/prijmač nastaví event stavovým strojom kanálov, že sa majú previesť do stavu monitoring, stavové stroje kanálov sa prevedú do stavu monitoring, čo predstavuje nastavenie continuous módu ADC prevodníkov, ktoré začnú vzorkovať kanály a nastavenie watchdogov na úrovne poslané z GUI, ak nastane prerušenie daného watchdogu, nastavia sa časovače, DMA, AD prevodník a spustí sa meranie, stavový stroj kanálu prejde do stavu \textit{MEASURING}, po ukončení merania DMA generuje prerušenie TC, v ktorom sa pošle event stavovému stroju vysielač/prijmač, že má poslať nové dáta, ten reaguje a prejde do stavu \textit{GATHERING\_TRANSFORMING\_SENDING} zo stavu \text{IDLE} a pošle označené dáta, keďže dáta kanálov sa posielajú zvlášť.
			
			\subsection{Postup v prípade automatického aktualizácie parametrov transformácie}
			 V prípade, že ide o požiadavku na zaslanie dát s inou transformáciou, teda o tzv. \text{ONLY\_TRANSFORM} typ správy, tak sa len prepočítajú parametre transformácií jednotlivých kanálov a registruje sa požiadavka na odoslanie oboch kanálov, stavový stroj prejde do stavu \textit{GATHERING\_TRANSFORMING\_SENDING}, v ktorom sa vykoná transformácia posledného merania a aktivuje sa DMA kanál pripojený na vysielač UART-u, ktorý pošle jednotlivé transformované merania do GUI.
			 
			 
		\end{multicols*}
		%\subfile{programCelkovaSchema}
\end{document}