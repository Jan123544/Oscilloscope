\documentclass[main.tex]{subfiles}
\begin{document}
	Okrem tejto transformácie je potrebné ešte prepočítať prahové napätie z voltov na úrovne príslušných watchdogov. Tento prepočet realizujeme vzorcom \cref{eqn:transformaciaThreshold}.
	\begin{equation}
	 	t = \floor{\frac{t_g}{r_m} \frac{t_{max}}{\alpha}}
	\label{eqn:transformaciaThreshold}
	\end{equation}
	kde $t$ je hodnota, ktorá sa zapisuje do threshold registrov watchdogov, $t_g$ je prahová hodnota napätia zaslaná z uživateľského prostredia, daná vo $[V]$, $t_{max}$ ja maximálna hodnota threshold registra pre daný watchdog ($2^{12}-1$ pre AWD1 a $2^{8}-1$ pre AWD2).
	
	Tiež musíme vypočítať parametre časovačov pomocou \cref{eqn:transformaciaCasovace}.
	\begin{equation}
		\begin{split}
				psc &= \floor{\frac{c}{c_{max}}} \\
				arr &= \floor{\frac{c}{psc + 1}}-1 \\
		\end{split}
		\label{eqn:transformaciaCasovace}
	\end{equation}
	kde $c$ je celkový počet taktov časovača obsiahnutých v nastavovanom časovom intervale v $[sec]$. Hodnotu $c$ vypočítame pre časovače s taktom $f$HZ na základe \cref{eqn:transformaciaPocetCyklov}.
	\begin{equation}
		c = \frac{f t_{pd} d }{n}
		\label{eqn:transformaciaPocetCyklov}
	\end{equation}
	kde $t_{pd}$ je časová základňa v jednotkách $\frac{sec}{dielik}$, $d$ je počet dielikov, $n$ je počet pretečení, ktoré majú za čas $t_{pd}d$ nastať.
\end{document}