\documentclass[main.tex]{subfiles}
\begin{document}
	Uživateľské prostredie očakáva dáta v normalizovanom formáte. Tento fromát je prednastavený na $4095$ úrovní. Kde $0$ znamená zakreslenie na najnižšiu úroveň v grafickom prostredí a $4095$ na najvyššiu. Pri posielaní dát to uživateľského rozhrania musí mikropočítač brať do úvahy nastavenia, ktoré si uživateľ zvolil a podľa toho transformovať body merania. Táto transformácia prebieha na základe \cref{eqn:transformacia}. 
	
	\begin{equation}
		v_n = \ceil{\floor{\frac{1}{s}v_\alpha + o}^{4095}}_{0}
		\label{eqn:transformacia}
	\end{equation}
	kde $v_n$ je hodnota posielaná do uživateľského prostredia, $s$ je prepočítaná citlivosť daná rovnicou \cref{eqn:transformaciaCitlivost}, $o$ je prepočítaný posun daný rovnicou \cref{eqn:transformaciaPosun} a $\floor{}^{4095}, \ceil{}_{0}$ sú funkcie definované podľa \cref{eqn:transformaciaCeilFloor}. Hodnota $v_{\alpha}$ je daná \cref{eqn:transformaciaAlpha}, kde $\alpha$ slúži na kalibráciu hodnoty $v_{adc}$ prevedenú AD prevodníkom.
	
	\begin{equation}
		s = \frac{s_gd_g}{r_m}
		 \label{eqn:transformaciaCitlivost}
	\end{equation}
	\begin{equation}
		o = o_g\frac{1}{s_gd_g}4095
		\label{eqn:transformaciaPosun}
	\end{equation}
	\begin{equation}
		\begin{split}
			\floor{x}^{4095} &= \begin{cases}
				x & \text{ ak } x < 4095 \\
				4095 & \text{ inak } \\
			\end{cases} \\
			\ceil{x}_{0} &= \begin{cases}
			x & \text{ ak } x > 0 \\
			0 & \text{ inak } \\
			\end{cases} \\
		\end{split}
	\label{eqn:transformaciaCeilFloor}
	\end{equation}
	\begin{equation}
 	v_{\alpha} = v_{adc} \alpha
	\label{eqn:transformaciaAlpha}
	\end{equation}
	kde hodnoty $s_g$, $d_g$ sú hodnoty citlivosti a posunu poslané z uživateľského prostredia v jednotkách $[\frac{V}{dielik}]$ a $[V]$. Parameter $r_m$ je rozsah na ktorom boli dáta merané teda $r_m \in \{5,10,20\}$.
	
	
	
\end{document}